\documentclass[a4paper,10pt]{article}
\usepackage[utf8]{inputenc}

%opening
\title{JobAdder: Requirements}
\author{}

\begin{document}

\maketitle

\section{Mandatory}
\subsection{General}
\begin{enumerate}
 \item JA must follow an object-oriented design.
 \item JA must be implemented in Python 3.
 \item All JA references and methods must have consistent Python type hints.
 \item JA must provide a central server that assigns user jobs to work machines.
 \item JA must provide a user client (UC) that transmits a user's job request to the central server.
 \item JA must provide a work machine client (WMC) that receives jobs from the central server and runs them on a work machine.
 \item All components of JA must work under the following Linux distributions: Ubuntu 18.04, CentOS 7.
 \item Users must be notified when one of their jobs terminates.
 \item Users must be able to query the state of jobs and work machines via a web interface or API.
\end{enumerate}
\subsection{User Client}
\begin{enumerate}
 \item UCs must be accessible through a CLI.
 \item UCs must provide to the user the same functionality as if the user was working directly on a work machine.
 \item UCs must allow the user to specify an operating system on which the job should be run.
 \item UCs must allow the user to specify minimum hardware requirements for a job: RAM, CPU thread count, GPU memory/model/speed.
 \item UCs must allow the user to specify a priority for a job.
\end{enumerate}
\section{Optional}
\begin{enumerate}
 \item Blocking mode: wait for the job to finish before client returns.
 \item Path translation: handle different mount point locations.
 \item Attach to stdout of running job or read log output (something like "tail -f").
 \item Estimate ETA for running jobs.
\end{enumerate}

\end{document}
