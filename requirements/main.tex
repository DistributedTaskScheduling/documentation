\documentclass[a4paper,10pt]{article}
\usepackage[utf8]{inputenc}
\usepackage{graphicx}       % provides commands for including figures


%opening
\title{JobAdder: Requirements}
\author{}

\begin{document}

\maketitle

\section{Mandatory}
\subsection{General}
\begin{enumerate}
 \item JA must follow an object-oriented design.
 \item JA must be implemented in Python 3.
 \item All JA references and methods must have consistent Python type hints.
 \item JA must provide a central server that assigns user jobs to work machines.
 \item JA must provide a user client (UC) that transmits a user's job request to the central server.
 \item JA must provide a work machine client (WMC) that receives jobs from the central server and runs them on a work machine.
 \item All components of JA must work under the following Linux distributions: Ubuntu 18.04, CentOS 7.
 \item Users must be notified when one of their jobs terminates.
\end{enumerate}
\subsection{Central Server}
\begin{enumerate}
 \item CSs must be accessible through a CLI.
 \item CSs must store data on queued/running/past jobs in a database.
 \item CS job database must not change on CS crash/restart.
 \item Users must be able to query the state of jobs and work machines from a CSs via a web interface or API.
\end{enumerate}
\subsection{User Client}
\begin{enumerate}
 \item UCs must be accessible through a CLI.
 \item UCs must provide to the user the same functionality as if the user was working directly on a work machine.
 \item UCs must allow the user to specify an operating system on which the job should be run.
 \item UCs must allow the user to specify minimum hardware requirements for a job: RAM, CPU thread count, GPU memory/model/speed.
 \item UCs must allow the user to specify a priority for a job.
\end{enumerate}
\section{Work machine client(WC)}
  \subsection{Functional requirements}
    \subsubsection{Main features}
    \begin{enumerate}
      \item[WCF10] On startup, WC shall connect to the server and report its hardware capabilities.
      \item[WCF20] WC must be able to receive commands from the central server.
      \item[WCF30] Each work machine must have a single instance of WC running.
      \item[WCF40] WC shall support the following commands:
      \begin{enumerate}
        \item[WCF41] Start/Resume execution of a job.
        \item[WCF42] Pause execution of a job.
        \item[WCF43] Remove a job from the list of running jobs on the current machine.
        \item[WCF44] Query the current state of the work machine: running jobs, load on hardware.
        \item[WCF45] Terminate this instance of the WC.
      \end{enumerate}
      \item[WCF50] WC shall download the job data to the local storage if needed. (TODO: need to decide when)
      \item[WCF60] WC shall notify the server whenever a job is completed.
    \end{enumerate}

    \subsubsection{Optional features}
      \begin{enumerate}
        \item[WCFO10] WC shall notify the server periodically on the progress of the job.
        \item[WCFO20] WC shall decide what kind of local storage to use for each job. (RAM, SSD or HDD).
        \item[WCFO30] WC shall estimate the time of arrival of a job.
      \end{enumerate}

  \subsection{Product Data}
  \begin{enumerate}
    \item[WCPD10] Docker containers for all the jobs running on this machine.
    \item[WCPD20] Results of the jobs running on this machine.
  \end{enumerate}

\section{Optional}
\begin{enumerate}
 \item Blocking mode: wait for the job to finish before client returns.
 \item Path translation: handle different mount point locations.
 \item Attach to stdout of running job or read log output (something like "tail -f").
 \item Estimate ETA for running jobs.
\end{enumerate}

\end{document}
