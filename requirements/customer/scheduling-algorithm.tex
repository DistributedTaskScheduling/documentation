\subsection{Scheduling Algorithm}
This section describes how the central server distributes jobs to work machines: the \textit{scheduling algorithm}.
Put briefly, the scheduling algorithm takes a list of jobs and decides which ones to run on the available work machines.
The scheduling algorithm has the following goals:
\begin{itemize}
\item Keep work machines busy: there should be no idle work machines if there are pending jobs that could be run on them.
\item Consider job priorities: jobs with high priorities should be worked on first.
\item Consider time of job submission: jobs that were submitted first should be worked on first (if priorities are equal).
\end{itemize}
The scheduling algorithm operates under the following conditions:
\begin{itemize}
\item Jobs arrive at unforeseen times.
\item Time required to complete jobs is unknown.
\item Jobs cannot be transferred from one work machine to another.
\item Jobs cannot be canceled and restarted.
\item Job preemption is limited: if a job is paused it has to be kept in memory.
Note that memory includes swap files.
\item Some jobs cannot be paused due to software licensing restrictions.
\end{itemize}
The scheduling algorithm takes the following parameters as input:
\begin{itemize}
\item List of pending jobs.
\item List of running/paused jobs per work machine.
\item Amount of work machine specific resources: CPU threads, memory, etc.
These resources can typically be freed by pausing jobs.
\item Amount of work machine unspecific resources: software licenses.
Whether these resources can be freed by pausing jobs depends on the software.
\item Size of work machine swap files.
This resource is used exclusively for holding paused jobs.
It has no influence on running jobs.
\end{itemize}
The scheduling algorithm reevaluates which jobs to run every time a job finishes running or when a new job request arrives.
The scheduling algorithm assigns jobs to work machines as follows:
\begin{enumerate}
\item If there are pending urgent jobs:
Assign oldest urgent jobs to work machines until urgent jobs or resources run out.
Non-urgent running jobs will be paused if possible and necessary to run more urgent jobs.
\item If there are paused high-priority jobs:
Assign oldest paused high-priority jobs to work machines until paused high-priority jobs or resources run out.
Running jobs with equal or lower priority will be paused if possible and necessary to run more old paused high-priority jobs.
\item Repeat 2. for paused medium-priority jobs.
\item Repeat 2. for paused low-priority jobs.
\item If there are pending high-priority jobs:
Assign oldest paused high-priority jobs to work machines until paused high-priority jobs or resources run out.
Running jobs with equal or lower priority will be paused if possible and necessary to run the oldest available high-priority job.
\item Repeat 5. for pending medium-priority jobs.
\item Repeat 5. for pending low-priority jobs.
\end{enumerate}