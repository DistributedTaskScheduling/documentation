\chapter{Purpose}
  \section{Product Goal}
    JobAdder aims to facilitate and accelerate the execution of computationally
    expensive jobs in a cluster of work machines.

    Currently, users need to access every single work machine manually in order to
    distribute their jobs over them. Even more problems arise when a group of users
    is involved. In that case, users have to look for the work machine with the
    least amount of work and add their job to it.

    JobAdder addresses and solves these problems. With JobAdder, users are able to
    send their jobs to a central server in order to evenly distribute them over
    suitable work machines without having to look for them manually. So if there is
    enough workload (which is usually the case), no machine remains idle unless it
    is forced to.

    Jobadder also aims to make task scheduling more user-friendly. A user does not
    have to keep checking the state of their job, they will be notified once the job
    is done. Nor does a user have to wait for an idle machine in order to queue a
    job, the central server saves them up and runs them once it is possible. There
    is also more insight into the whole system by providing statistics about user
    activity, currently running jobs, past jobs and queued jobs.

    Users can set priorities for their jobs so that jobs with high priority will be
    executed before jobs with low priority. Urgent jobs are immediately executed by
    pausing jobs with the lowest priority, so a user is not required to pause or
    cancel low-priority jobs manually. The scheduling algorithm also guarantees that
    every job is eventually run by automatically increasing its priority over time.

  \section{Target Audience}
    The target audience mostly consists of groups of users with access to several
    work machines. This includes companies, academic institutions and computing
    centers. These users usually need remote machines to run their jobs due to high
    hardware requirements.

    JobAdder helps them use their hardware more efficiently by adding an abstraction
    layer between the user and the work machines that are running the user's jobs.
