\newglossaryentry{ID}
{
  name=ID,
  plural=IDs,
  description={
    An ID is a string of digits, english letters (both lower- and uppercase) and slashes '-'.
  }
}
\newglossaryentry{job status}
{
  name=job status,
  plural=job status,
  description={
    The status of a job can be:
    \begin{itemize}
      \item Queued - the job is waiting for a suitable work machine to become available.
      \item Running - the job is running on some work machine.
      \item Paused - the job has been running, but has been paused to allow for execution of other jobs.
      \item Done - the job has finished running.
      \item Killed - the job was forcefully stopped.
      \item Crashed - the work machine the job was running on shut down unexpectedly.
    \end{itemize}
  }
}
\newglossaryentry{job priority}
{
  name=job priority,
  plural=job priorities,
  description={
    The priority of a job can be one of the following (in ascending order):
    \begin{itemize}
      \item Low
      \item Medium
      \item High
      \item Urgent
    \end{itemize}
  }
}
\newglossaryentry{job}
{
  name=job,
  plural=jobs,
  description={
      A job in the context of JobAdder consists of the following:
      \begin{itemize}
        \item A unique job \gls{ID}.
        \item A program to be executed.
        \item A unix user who owns the job.
        \item \Gls{job priority}.
        \item Number of CPU threads needed for the execution of the program, an integer.
        \item Amount of RAM in megabytes needed for the execution of the program, an integer.
        \item Paths in the network storage needed for the application.
        \item One of:
        \begin{itemize}
          \item A Dockerfile providing instructions to build program environment.
          \item The name of predefined container environment (TODO: link to list of environments we want to provide, in the WMC)
        \end{itemize}
     \end{itemize}
  }
}
\newglossaryentry{timestamp}
{
  name=timestamp,
  plural=timestamps,
  description={
    A timestamp is a combination of the date and time of a particular point in time.
    It has the following format: "YYYY-DD-MM HH:MM:SS".
  }
}
\newglossaryentry{administrator}
{
  name=administrator,
  plural=administrators,
  description={
    An administrator is a user who can terminate queued and running jobs from all users.
  }
}
\newglossaryentry{UID}
{
  name=UID,
  plural=UIDs,
  description={
    A UID (user identifier) is a number assigned by Linux to each user on the system.
  }
}
\newglossaryentry{GID}
{
  name=GID,
  plural=GIDs,
  description={
    A GID (group identifier) is a number assigned by Linux to each group on the system.
  }
}
\newglossaryentry{Dockerfile}
{
  name=Dockerfile,
  plural=Dockerfiles,
  description={
     A Dockerfile is a text document that contains all the commands a user could call on the
     command line to assemble an image.
  }
}
