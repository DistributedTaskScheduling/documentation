\chapter{Product requirements}
\section{Customer stipulated functionality}
\begin{enumerate}
  \item JA shall provide a user client which makes it easy for users to schedule new jobs.
  \item For each job, the user shall be able to set the following attributes via the user client:
    \begin{itemize}
      \item Target OS.
      \item Minimum amount of RAM.
      \item Minimum number of CPU threads.
      \item Job priority.
    \end{itemize}
  \item Users shall be able to receive notifications when their job is completed.
  \item JA must provide a central server that assigns user jobs to work machines.
  \item The central server shall store information about all jobs in a persistent database.
  \item The central server shall provide a network API to get statistics about CPU load, number of queued/running jobs, etc.
  \item JA must provide a worker client that receives jobs from the central server and runs them on a work machine.
  \item All components of JA must work under the following Linux distributions: Ubuntu 18.04, CentOS 7.
\end{enumerate}
\section{Technical requirements}
\begin{enumerate}
  \item JA must follow an object-oriented design.
  \item JA must be implemented in Python 3.
  \item All JA references and methods must have consistent Python type hints.
\end{enumerate}
\section{Developed contributions}
\begin{enumerate}
  \item Job preemption: if higher priority jobs are scheduled, the existing jobs may be preempted to run the high priority jobs.
  \item Blocking mode: wait for the job to finish before client returns.
  \item Path translation: handle different mount point locations.
  \item Attach to stdout of running job or read log output (something like "tail -f").
  \item Specify custom docker containers to execute.
  \item Configuration files for reusing configuration of jobs which are run often with the same parameters.
  \item SSH access to a running job.
\end{enumerate}
