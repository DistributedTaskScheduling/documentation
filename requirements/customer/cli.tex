\newcommand{\jacommand}[3]{

\textsc{\large{\textbf{#1}}}
\begin{itemize}
\item[\textbf{Options:}] #2
\item[\textbf{Effect:}] #3
\end{itemize}
}
\newcommand{\jaoption}[3]{
\begin{itemize}
\item[] \textbf{#1} (#2)
\item[\textbf{Effect:}] #3
\end{itemize}
}
\subsection{Command Line Interface}
All components of JobAdder are accessible through a command line interface (CLI).
The command line interface on a machine with JobAdder installed has the following structure:
\begin{equation}
\text{<component name> <command> <options> <targets>},
\end{equation}
where the \textit{component name} can be any one of the following:
\begin{itemize}
\item \textbf{jobadder} to access the \textit{user client}.
\item \textbf{jobcenter} to access the \textit{server}.
\item \textbf{jobworker} to access the \textit{worker client}.
\end{itemize}
Commands, options, and valid targets are and options are specified in the following sections.

Options can be added in the following ways:
\begin{itemize}
\item By supplying their names like "--<option\_x> --<option\_y>=2 --<option\_z>".
%\item By supplying their abbreviations like "-x -y 2 -z".
%\item By supplying their abbreviations like "-xz -y 2".
\end{itemize}

The available \textit{commands} are generally specific to the used component.
Likewise, the available \textit{options} and \textit{targets} are generally specific to the commands of a used component.
However, some commands and options are shared between components or commands.

The JobAdder CLI ignores capitalization.
JobAdder CLI options override the values of configuration files.
If no command is given, the general help text for the given component is printed out.
\subsubsection{Shared Options}
\jaoption{config}{c}{Overrides the global configuration file with the given configuration file.}
\jaoption{detach}{d}{Detaches the execution of the given command from the command line.}
\jaoption{help}{h}{
Prints out a list of the available commands. 
When used in conjunction with a command:
prints out more detailed information on the given command and a list of the available options.}
\jaoption{verbose}{v}{
Determines the amount of information printed out.
If set to 2, detailed information is printed out.
If set to 1, only high-level information is printed out (default).
If set to 0, no information is printed out.}
\subsubsection{User Client}
\jacommand{add}
{gpu, mount, name, priority, server, ram, threads}
{Sends one or more job requests to the server.
If targets are Docker containers the worker client simply runs the container.
If targets are configuration files with a specified terminal command the worker executes the given terminal command.
Otherwise targets are interpreted as terminal commands}
\jaoption{gpu}{g}{
If set to true, informs the server that the job requires a gpu.
If set to a string that can be interpreted as VRAM amount, informs the server that a GPU with this amount of VRAM is required for the job.
If set to a specific GPU model, informs the server that the job must be run on a machine with that GPU built in.
}
\jaoption{mount}{m}
{Directories that need to be mounted onto the Docker container that the job is running in.}
\jaoption{name}{n}
{Human-readable name of the job.
Defaults to user's name + job index}
\jaoption{priority}{p}
{Sets the priority of the job: Very high, high, medium, or low.
Defaults to medium.}
\jaoption{server}{s}
{Server to send the job request to.
Defaults to localhost.}
\jaoption{ram}{r}
{If set to a string that can be interpreted as RAM amount, informs the server that this amount of RAM is required for the job.}
\jaoption{threads}{t}
{If set to an integer, informs the server that this number of threads is required for the job.}

\jacommand{pause}{None}{Pauses the job with the specified name.}

\jacommand{query}{user}
{Prints out information on running/queued/past jobs (specified by target).
In addition to the one below, options from add command can be used as constraints.}
\jaoption{after}{a}{Only return jobs added after the given point in time.}
\jaoption{before}{b}{Only return jobs added before the given point in time.}
\jaoption{user}{u}{Filter query by user that added the job.}

\jacommand{resume}{None}{Resumes the job with the specified name.}

\jacommand{stop}{None}{Stops the job with the specified name.}

\subsubsection{Server}
\jacommand{start}{None}{Starts an instance of the JobAdder server.}
\jacommand{stop}{None}{Stops instances of the JobAdder server.}

\subsubsection{Worker Client}
\jacommand{start}{None}{Starts an instance of the JobAdder worker client.}
\jacommand{stop}{None}{Stops instances of the JobAdder worker client.}
