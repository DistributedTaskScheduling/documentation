\section{Functional Requirements}
\begin{itemize}
  \item [UCF01] Queueing new jobs
  \item [UCF02] Querying the state of jobs
  \item [UCF03] Canceling jobs
  \item [UCF04] Setting job priorities
  \item [UCF05] Checking the specifications of available containers
  \item [UCF06] Choosing the parameters of the container(preferred OS/environment)
  \item [UCF07] Choosing the running mode: "blocking mode"/ default mode
  \item [UCF08] Choosing the minimal hardware requirements for a job
  \item [UCF09] Checking the state of the entire server
\end{itemize}

\section{Optional Requirements}
\begin{itemize}
  \item [UCOR01] Saving a private image for future usage
  \item [UCOR02] Scheduling a job to run at a specific time
  \item [UCOR03] Checking the estimated waiting time
  \item [UCOR04] Checking the estimated time of arrival
\end{itemize}

\section{Requirement Details}
\begin{itemize}
  \item UCFR01 Queueing new jobs
    \begin{itemize}
      \item Goal: Allow User to queue jobs.
      \item Importance: Primary
      \item Precondition: User is logged into a user account.
      \item Post condition (success): The job is queued.
      \item Post condition (fail):
        \begin{itemize}
          \item The job is not queued.
          \item An appropriate error message is displayed.
        \end{itemize}
      \item Triggering Event: The User requests to add a job to the queue.
      \item Description:
        \begin{itemize}
          \item  The user can add a job by calling the "add" command of the CLI of the user client.
        \end{itemize}
    \end{itemize}

  \item UCFR02 Querying the state of jobs
    \begin{itemize}
      \item Goal: Allow User to check the state of a job.
      \item Importance: Primary
      \item Precondition: At least one job must be in the central server's job database.
      \item Post condition (success): A message with the state of the jobs will be displayed.
      \item Post condition (fail):
        \begin{itemize}
          \item The central server can't find the specified job.
          \item An appropriate error message is displayed.
        \end{itemize}
      \item Triggering Event: The user queries the state of a job with a given job name.
      \item Description:
        \begin{itemize}
          \item The user can check the state of a job by calling the "query" command of the CLI of the user client.
        \end{itemize}
    \end{itemize}

  \item UCFR03 Canceling jobs
    \begin{itemize}
      \item Goal: Allow User to cancel a running Job.
      \item Importance: Primary
      \item Precondition: The job must be running or in the queue.
      \item Post condition (success):
        \begin{itemize}
          \item The job is removed from the queue or from the working machine and its resources are released.
          \item An appropriate success message is displayed.
        \end{itemize}
      \item Post condition (fail):
        \begin{itemize}
          \item The job is still running or in the queue.
          \item An appropriate error message is displayed.
        \end{itemize}
      \item Triggering Event: The user cancels a running job or a queued job.
      \item Description:
        \begin{itemize}
          \item   The user can cancel a job by calling the "stop" command of the CLI of the user client.
        \end{itemize}
    \end{itemize}

  \item UCFR04 Setting job priorities
    \begin{itemize}
      \item Goal: User sets the priority of a specific job.
      \item Importance: Primary
      \item Precondition: User must be logged in.
      \item Post condition (success):
        \begin{itemize}
          \item Job priority changes.
          \item An appropriate success message is displayed.
        \end{itemize}
      \item Post condition (fail):
        \begin{itemize}
          \item Invalid priority.
          \item An appropriate error message is displayed.
        \end{itemize}
      \item Triggering Event: The user sets the priority of a job.
      \item Description:
        \begin{itemize}
          \item The user adds the priority of the job while adding it to the queue with the "add command".
          \item The user changes the priority of a job after it already start running with the "set" command.
        \end{itemize}
    \end{itemize}


  \item UCFR05 Checking the specifications of available containers
    \begin{itemize}
      \item Goal: User receives a description of the available containers.
      \item Importance: Primary
      \item Precondition: User must be logged in.
      \item Post condition (success): A description of the available containers is displayed.
      \item Post condition (fail):
        \begin{itemize}
          \item Networking error.
          \item An appropriate error message is displayed.
        \end{itemize}
      \item Triggering Event: The user requests containers specifications.
      \item Description:
        \begin{itemize}
          \item The user requests containers specifications with the "query" command.
        \end{itemize}
    \end{itemize}

  \item  UCFR06 Choosing the parameters of the container(e.g. base image)
    \begin{itemize}
      \item Goal: User chooses preferred container parameters for a specific job.
      \item Importance: Primary
      \item Precondition: User must be logged in.
      \item Post condition (success): An appropriate success message is displayed.
      \item Post condition (fail):
        \begin{itemize}
          \item unavailable specifications.
          \item An appropriate error message is displayed.
        \end{itemize}
      \item The user chooses container parameters.
      \item Description:
        \begin{itemize}
          \item The user sets container parameters when calling the "add" command of the CLI of the user client.
        \end{itemize}
    \end{itemize}

  \item  UCFR07 Choosing the running mode
    \begin{itemize}
      \item Goal: User chooses preferred mode for the application to use (regular or blocking).
      \item Importance: Primary
      \item Precondition: User must be logged in.
      \item Post condition (success): An appropriate success message is displayed.
      \item Post condition (fail):
        \begin{itemize}
          \item unavailable mode.
          \item An appropriate error message is displayed.
        \end{itemize}
      \item Triggering Event: The user chooses a specific mode.
      \item Description:
        \begin{itemize}
          \item When the user chooses regular mode the user client will instantly return to the regular command line.
          \item When the user chooses blocking mode the user client will only return to the command line once the job has finished running.
        \end{itemize}
    \end{itemize}

  \item  UCFR08 Choosing the minimal hardware requirements for a job
    \begin{itemize}
      \item Goal: User chooses the minimal hardware required for the job to run.
      \item Importance: Primary
      \item Precondition: User must be logged in.
      \item Post condition (success): An appropriate success message is displayed.
      \item Post condition (fail):
        \begin{itemize}
          \item unavailable requirements.
          \item An appropriate error message is displayed.
        \end{itemize}
      \item Triggering Event: The user chooses the minimal hardware requirements for the job.
      \item Description:
        \begin{itemize}
          \item The user sets the minimum hardware requirements when calling the "add" command of the CLI of the user client.
        \end{itemize}
    \end{itemize}

  \item  UCFR09 Checking the state of the central server
    \begin{itemize}
      \item Goal: User requests a server state check.
      \item Importance: Primary
      \item Precondition: User must be logged in.
      \item Post condition (success): A description of the state of the server is displayed.
      \item Post condition (fail):
        \begin{itemize}
          \item Networking error.
          \item An appropriate error message is displayed.
        \end{itemize}
      \item Triggering Event: The user requests a server state check.
      \item Description:
        \begin{itemize}
          \item The user requests a server state check with the "query" command.
        \end{itemize}
    \end{itemize}
\end{itemize}
