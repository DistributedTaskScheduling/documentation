\section{Functional Requirements}
\begin{itemize}
  \item[UCF10] The user client shall implement a command line interface as specified in chapter \hyperref[CLI]{CLI}.
  \item[UCF20] In order to carry out a request, the user client shall authenticate the user with the central server as follows:
    \begin{enumerate}
      \item[UCF21] Establish a \gls{SSH} connection to the server using the user's credentials.
        Credentials are either stored in a local file, environment variable or are entered manually by the user.
      \item[UCF22] Once the user is authenticated, the actual request is sent to the server using the utility program described in CSF40.
    \end{enumerate}

  \item[UCF30] User client shall enable the user to submit new jobs to the central server using the `add` command.
    In case the authentication fails, the job shall not be queued and an error message shall be displayed.
    Otherwise, the new job shall be queued in the central server with all options set to the values specified from the user.
    After that, the user client shall exit, except if blocking mode is requested or if the user attaches his terminal to the job.

  \item[UCF40] User client shall enable the user to query the state of a job by the job's ID.
    In case the authentication fails, or the job is not found in the server database, an appropriate error message shall be displayed.
    Otherwise, the job's status shall be printed to the command line.

  \item[UCF50] User client shall enable the user to terminate jobs.
    In case the authentication fails, or the user does not have sufficient privileges to terminate the job,
    the job continues running, and an appropriate error message is displayed.
    Otherwise, the termination request shall be submitted to the central server.

  \item[UCF60] User client shall support adding a job in blocking mode.
    In blocking mode, instead of returning immediately after the job is scheduled, the user client shall remain running
    and shall print the stdout and stderr of the running job, as well as status messages whenever the job is started or paused.

  \item[UCF70] User client shall support querying the status of jobs of a given user.
    For each selected job, a line will be printed containing the following information:
    \begin{itemize}
      \item Job ID.
      \item Job submission time.
      \item Job status.
      \item Time of last status update.
      \item Total running time up to now.
      \item Work machine the job is scheduled on(if known).
    \end{itemize}
\end{itemize}

\section{Optional Requirements}
\begin{itemize}
  \item [UCO10] Users can save a docker container for future use.
  \item [UCO20] The user client shall implement an option to schedule a job at a specific time point in the future.
  \item [UCO30] The user client shall report the number of jobs at the queue when queueing a job, as well as a rough estimate for waiting time.
\end{itemize}


