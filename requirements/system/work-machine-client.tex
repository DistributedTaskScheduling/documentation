\chapter{Worker Client}
  This chapter gives a detailed description of the requirements for the worker client.
  \section{Functional Requirements}
    \subsection{Main Features}
    \begin{enumerate}
      \item[WCF10] On startup, WC shall connect to the server and report its hardware capabilities.
      \item[WCF20] WC must be able to receive commands from the central server.
      \item[WCF30] Each work machine must have a single instance of WC running.
      \item[WCF40] WC shall support the following commands:
      \begin{enumerate}
        \item[WCF41] Start/Resume execution of a job.
        \item[WCF42] Pause execution of a job.
        \item[WCF43] Remove a job from the list of running jobs on the current machine.
        \item[WCF44] Query the current state of the work machine: running jobs, load on hardware.
        \item[WCF45] Terminate this instance of the WC.
      \end{enumerate}
      \item[WCF50] WC shall download the job data to the local storage if needed. (TODO: need to decide when)
      \item[WCF60] WC shall notify the server whenever a job is completed.
    \end{enumerate}

    \subsection{Optional Features}
      \begin{enumerate}
        \item[WCFO10] WC shall notify the server periodically on the progress of the job.
        \item[WCFO20] WC shall decide what kind of local storage to use for each job. (RAM, SSD or HDD).
        \item[WCFO30] WC shall estimate the time of arrival of a job.
      \end{enumerate}

  \section{Product Data}
  \begin{enumerate}
    \item[WCP10] Docker containers for all the jobs running on this machine.
    \item[WCP20] Results of the jobs running on this machine.
  \end{enumerate}
